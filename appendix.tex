\newpage
\appendix
\part{Appendices}

\section{Honesty declaration}
Hereby, I declare that I have composed the presented work Lost in transition independently on my own and without any other resources than the ones indicated. All thoughts taken directly or indirectly from external sources are properly denoted as such.
\vspace{3cm}

\centering
\begin{tabular}{p{10mm}>{\centering\arraybackslash}p{50mm}p{10mm}
>{\centering\arraybackslash}p{50mm}p{10mm}}
&\textit{\large Windisch,}&&& \\
&\textit{\large  \today}&&\large Jonas Oesch& 
\end{tabular}
\\

\newpage
\begin{flushleft}
\section{Corpus of narrative visualizations} \label{appendix-corpus}

The collection of 147 narrative visualizations from three collections. The collection is available on GitHub \url{https://github.com/jonasoesch/lostintransition} in the folder \texttt{supplementary/B-corpus}.

\section{Analysis of narrative visualizations} \label{appendix-transition-analysis}

The anlysis of of a selection of narrative visualizations to identify different transition types. The analysis is available on GitHub \url{https://github.com/jonasoesch/lostintransition} in the folder \texttt{supplementary/C-analysis}.


\section{Stories} \label{appendix-story}

The mortality and energy story that were used in the experiment. The stories are linked on GitHub \url{https://github.com/jonasoesch/lostintransition} in the folder \texttt{supplementary/D-stories}.


\section{Experiment questionnaire} \label{appendix-questionnaire}

The questionnaire that was presented after each mini-story. The questinnaire is available on GitHub \url{https://github.com/jonasoesch/lostintransition} in the folder \texttt{supplementary/E-questionnaire}.


\section{Experiment survey} \label{appendix-survey}

The survey that was presented to the participants at the end of the experiment. The survey is available on GitHub \url{https://github.com/jonasoesch/lostintransition} in the folder \texttt{supplementary/F-survey}.


\section{Passive data collected in the experiment} \label{appendix-passivedata}

The schema for the data that was collected passively during the experiment. The schema is available on GitHub \url{https://github.com/jonasoesch/lostintransition} in the folder \texttt{supplementary/G-passive-data}.

\section{Data analysis} \label{appendix-dataanalysis}

The jupyter notebooks that were used for the data analysis. The analysis is available on Github \url{https://github.com/jonasoesch/lostintransition} in the folder \texttt{supplementary/H-analysis}.


\section{Implementation} \label{appendix-implementation}

An implementation of a program generator that interprets the described syntax and produces a narrative visualization with animated transitions. The implementation is linked on Github \url{https://github.com/jonasoesch/lostintransition} in the folder \texttt{supplementary/I-implementation}.


\newpage
\section{Design choices} \label{appendix-designchoices}
\ref{sec:narr-vis} and \ref{sec:comparison}) and the experiment did not provide an introduction to the topic. A story on paratransit (a special
means of transport for disabled people in the U.S.) for example, was initially planned to be included but was soon removed because the topic
was not familiar to most people.

The stories also had to avoid highly controversial topics because we feared that implicit reader knowledge might interfere with the
interpretation of the visualizations (see section \ref{sec:comparison}). We have seen this happen in both stories but it was less prevalent than we feared.

Finally, the story needed to be told mainly through the visualizations.
This turned out to be to most constraining factor. In most existing
narrative visualization we have found, that the textual narration was
essential to understand the story. The story on the evolution of
mortality was chosen exactly because it contained very little
text in its original version. The story about the energy sources was
specifically created in a way that we hoped would be self-explanatory
with very little text.

\hypertarget{minimal-textual-narrative}{%
\subsubsection{Minimal textual
narrative}\label{minimal-textual-narrative}}

Textual narrative was excluded from the experiment because it is a huge
confounder. In typical narrative visualization, the story is
presented through textual or audio narrative. The visualizations mostly
serve to reinforce the point. But, when presented like this, relating
the participants answers to any differences in the visualizations would
be very difficult. On the other hand, charts are impossible to interpret
without at least some text. We, therefore, decided to include labels as
well as a chart title. But we made sure that nothing was hinting at a
relationship in any of these. Each chart is completely self-contained
and provides interesting information even without the other chart.

\hypertarget{visualization-literacy}{%
\subsubsection{Visualization literacy}\label{visualization-literacy}}

\begin{figure}
\centering
\includegraphics{/Users/jonas/Desktop/P9/bericht/img/experiment-charttypes.png}
\caption{Different chart types and their difficulty according to our tests. \label{experiment-charttypes}}
\end{figure}

Previous studies have found that the majority of people are unable to
correctly interpret complex visualizations. (TODO:source) According to
these authors, the ``safe'' visualizations are bar charts, line charts,
scatterplots and maps (see section \ref{sec:vis-literacy}). This finding
is supported by our analysis of narrative visualization in section
\ref{sec:corpus-analysis} who also almost exclusively use these simpler
visualizations As we did not want participants to fail because they were
unable to read the individual charts, we have decided to limit the
experiment to line charts and stacked area charts. In our pilot studies,
these chart types have been ``safe'', even though the stacked area chart
posed problems to some of the participants in the experiment (see
\ref{sec:experiment-results}). One chart type that was excluded based on
pilot data was the \emph{slope chart}. (see \ref{experiment-charttypes}
)

\hypertarget{reader-controlled-animation}{%
\subsubsection{Reader-controlled
animation}\label{reader-controlled-animation}}

Another factor to consider was the amount of reader control or
interactivity. Prior work has demonstrated positive effects of animated
transitions often involved higher levels of interactivity compared to
the static transitions. (TODO:source) But interactivity has been shown
to have benefits in itself, for example for learning. (TODO:source) All
transitions were therefore designed to be totally reader controlled. By
scrolling down, readers could advance to the next chart, by scrolling
up, they could go back to the previous chart.

We also tried to avoid discussions about the proper duration of animated
transitions by making the animations completely controllable through
scrolling. Interestingly we have not found this to be common practice in
our analysis of narrative visualizations. The most prevalent case is a
fixed-duration animation that is triggered by scrolling to a certain
point.

\hypertarget{animation-design}{%
\subsubsection{Animation design}\label{animation-design}}

Three different kinds of animated transitions were used throughout the
experiment which were all concerned with supporting \emph{object
constancy} in different scenarios:

\hypertarget{one-to-many}{%
\paragraph{One-to-many}\label{one-to-many}}

A good example of this case is \emph{Mortality D}
(sec.~\ref{sec:mortality-d}). There, the animation needed to convey that
the ``causes of death'' \includegraphics{img/char.pdf} in the second
chart only concerned the ``25--44'' year old age group
\includegraphics{img/ctxt.pdf} . This is done through a \emph{staged
transition} (see section \ref{sec:staged-animation}). We first highlight
the ``25--44'' year old age group by hiding all the others and then
splitting and morphing this character into the different causes of
death. The same approach was applied in \emph{Mortality B}
(sec.~\ref{sec:mortality-b}), \emph{Energy B} (sec.~\ref{sec:energy-b})
and \emph{Energy C} (sec.~\ref{sec:energy-c}).

\hypertarget{many-to-one}{%
\paragraph{Many-to-one}\label{many-to-one}}

This case can only be found in \emph{Mortality A}
(sec.~\ref{sec:mortality-a}). Here the different age groups
\includegraphics{img/char.pdf} are being morphed into a single line that
represents ``Everyone'' \includegraphics{img/char.pdf}. After the
morphing was finished, the characters (``Men''
\includegraphics{img/char.pdf} and ``Women''
\includegraphics{img/char.pdf}) were shown.

\hypertarget{many-to-many}{%
\paragraph{Many-to-many}\label{many-to-many}}

This case is well illustrated by \emph{Energy D}
(sec.~\ref{sec:energy-d}) where the marks for proportions
\includegraphics{img/attr.pdf} of different energy sources
\includegraphics{img/char.pdf} were morphed to represent the price
evolution \includegraphics{img/attr.pdf} of these same energy sources
\includegraphics{img/char.pdf}. Many-to-many animations are also being
used in \emph{Mortality C} (sec.~\ref{sec:mortality-c}) and \emph{Energy
A} (sec.~\ref{sec:energy-a}).

\hypertarget{animated-axis-interpolation}{%
\subsubsection{Animated axis
interpolation}\label{animated-axis-interpolation}}

In the first pilot study, we have included axis interpolation for some
transitions. But they
were excluded from the final experiment as they introduced another
confounder and were not directly related to the research question. In
our analysis of transitions in narrative visualization, we have found
several different approaches to animating axis interpolation. A topic
that would certainly merit further research.

\hypertarget{interpolation}{%
\subsubsection{Interpolation}\label{interpolation}}

\begin{figure}
\centering
\includegraphics{/Users/jonas/Desktop/P9/bericht/img/Interpolation.png}
\caption{Different color interpolation mehthods compared. \label{design-interpolation}}
\end{figure}

For all interpolations, we implemented ``slow-in-slow-out''-easing. For interpolating between colors, we used a
perceptually uniform HCL-interpolation because it ``intuitively looks
right''. Notice in figure \ref{design-interpolation} how RGB and
LAB tend to desaturate while HSL and CubeHelix tend to oversaturate, HCL
strikes a good balance.
\end{flushleft}
\newpage


\vspace*{\fill}
The End.
\vspace*{\fill}
